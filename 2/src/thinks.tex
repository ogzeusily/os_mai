
\vspace{15\baselineskip}
\section{Выводы}

В одном цифровом королевстве жил да был Программист по имени Артем. Писал он программу для Великого Моделирования, но была она медленной, как улитка, ибо делала все дела по очереди, одним потоком.

Решил Артем ускорить ее, создав целую команду потоков-работников. Но вышла суматоха! Работники наперегонки кидались к общим сундукам с данными, переписывали друг другу результаты, и в итоге Великое Моделирование выдавало каждый раз новый, неверный ответ. Сундуки ломались, а данные рассыпались в прах. Артем был в отчаянии.

Однажды мудрый Системный Архитектор дал ему древний свиток под названием «Лабораторная работа: Мьютексы и Потоки». Артем уединился и погрузился в чтение.

И озарение снизошло на него! Он понял, что его работникам не хватает не скорости, а порядка. Он осознал:

1.  Мьютекс — это ключ. Он создал для каждого сундука с общими данными волшебный ключ-мьютекс. Прежде чем взять или положить что-либо, поток-работник должен был взять этот ключ. Занятый ключ означал, что сундуком кто-то пользуется, и другие терпеливо ждали своей очереди.

2.  Порядок — основа скорости. Теперь, защитив общие данные, его потоки перестали мешать друг другу. Они слаженно трудились над своими частями задачи, а когда нужно было поделиться результатом — использовали ключ. Суматоха сменилась гармоничной работой.

3.  Код — это крепость. Артем вынес все магические заклинания для приручения потоков в отдельную башню — os\_linux.c. Он с гордостью думал, что если королевству придется переехать в другую операционную систему, ему нужно будет перестроить лишь эту одну башню, а не всю крепость.

С новыми знаниями Артем переписал свою программу. Запустил... и она заработала! Быстро, без ошибок, выверенно. Потоки-работники трудились в идеальной синхронизации, а Великое Моделирование наконец-то показывало верный и стабильный результат.

С тех пор Артем больше никогда не запускал потоки без мьютексов. И в его коде всегда царили порядок и предсказуемость.
\pagebreak
