\section{Метод решения}
В Unix-подобных системах используются стандартные средства POSIX Threads (Pthreads): pthread\_create, pthread\_join, а также примитивы синхронизации pthread\_mutex\_t. Для переносимости создана абстракция, состоящая из заголовочного файла os.h и его реализации для конкретной ОС (linux/os\_linux.c).

Программа запускает пул рабочих потоков для параллельного выполнения симуляций (экспериментов). Общее количество экспериментов распределяется между доступными потоками. Каждый поток выполняет свою часть симуляций и атомарно обновляет общие счетчики побед, используя мьютекс для предотвращения состояния гонки.

\section{Описание программы}

\texttt{main.c} --- точка входа в программу. Отвечает за парсинг аргументов командной строки, инициализацию общих данных (параметры игры, счетчики побед) и мьютекса. Создает и запускает рабочие потоки, распределяя между ними нагрузку. После завершения всех потоков вычисляет и выводит итоговые шансы на победу.

\texttt{os.h} --- объявление общего кроссплатформенного интерфейса для работы с потоками и мьютексами. Содержит объявления типов OsThread, OsMutex и функций-оберток.

\texttt{child2.c} --- исполняемый файл второго дочернего процесса. Удаляет повторяющиеся пробелы из входного потока (оставляет только один пробел подряд) и выводит результат в стандартный поток вывода.

\texttt{os\_linux.h} --- заголовочный файл для реализации под Linux. Включает <pthread.h> и определяет OsThread как pthread\_t, а OsMutex как pthread\_mutex\_t.  
\texttt{os\_linux.c} --- реализация функций-оберток над системными вызовами POSIX Threads для управления потоками и синхронизации. 

\vspace{1\baselineskip}
Основные функции:
\begin{itemize}
    \item \texttt{OsThreadHandle os\_thread\_create(void *(*start\_routine)(void *), void *arg);} --- создаёт новый поток. Обёртка над \texttt{pthread\_create()}.
    \item \texttt{int os\_thread\_join(OsThreadHandle thread);} --- ожидает завершения указанного потока. Обёртка над \texttt{pthread\_join()}.
    \item \texttt{int os\_mutex\_init(OsMutex *mutex);} --- инициализирует мьютекс. Обёртка над \texttt{pthread\_mutex\_init()}.
    \item \texttt{int os\_mutex\_lock(OsMutex *mutex);} --- блокирует мьютекс. Обёртка над \texttt{pthread\_mutex\_lock()}.
    \item \texttt{int os\_mutex\_unlock(OsMutex *mutex);} --- разблокирует мьютекс. Обёртка над \texttt{pthread\_mutex\_unlock()}.
    \item \texttt{int os\_mutex\_destroy(OsMutex *mutex);} --- уничтожает мьютекс. Обёртка над \texttt{pthread\_mutex\_destroy()}.
    \item \texttt{int os\_get\_cpu\_count(void);} --- возвращает количество доступных ядер процессора. Обёртка над \texttt{sysconf(\_SC\_NPROCESSORS\_ONLN)}.
\end{itemize}

\vspace{2\baselineskip}

Главный поток (main) подготавливает структуру ThreadData, содержащую параметры игры и указатели на общие счетчики побед. Затем он создает N рабочих потоков, где N равно количеству доступных ядер ЦП. Каждый поток получает указатель на ThreadData и выполняет функцию experiment\_routine. В этой функции каждый поток симулирует свою порцию игровых партий. По завершении каждой симуляции поток захватывает мьютекс, чтобы безопасно увеличить счетчик побед одного из игроков, и сразу же освобождает его. Главный поток ожидает завершения всех рабочих потоков с помощью os\_thread\_join, после чего выводит итоговую статистику.